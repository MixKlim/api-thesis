% !TEX root = ../tex/thesis.tex

\chapter{Introduction}
\label{ch:intro}

% Define the location of your plots
\graphicspath{{./../gfx/Intro/}}

% Main body
This file contains a thesis template that should help get you started designing your thesis. It is based on the theses of Nathalie Degenaar and Gijs Mulders.

\section{An example section}
\label{ch1:sec:example}
    You can reference a section in this chapter (Sec. \ref{ch1:sec:example}),
    using \verb|Sec. \ref{}|. Alternatively, you can reference to chapter 
    \autoref{ch:paper1} or a section there (\autoref{ch:p1:data}) using the
    \verb|\autoref{}| command.  
    It's good practice to label chapters with \verb|\label{ch1:}|, section with
    \verb|\label{sec:}|, figures with \verb|\label{fig:}|, equations with
    \verb|\label{table:}| and equations with \verb|\label{eq:}|. This avoids
    double labling in a long document like a thesis. 

\subsection{A subsection}

\begin{figure}[h]
    \centering
    \includegraphics[width=0.8\linewidth]{{figure1.eps}}
    \caption[]{Number of exoplanets discovered each year since 1995. Data taken from exoplanets.eu. 
        \label{ch1:fig:figure1}}
\end{figure}

It was only eighteen years ago that the first planet orbiting another star was discovered by \cite{1995Natur.378..355M}, and the number of exoplanets found has increased ever since (Figure \ref{ch1:fig:figure1}).

\section{Summary: thesis outline}
You can put an English summary here (don't forget to keep the word summary in the title or the pedel might complain!), or in the end with the dutch summary.

