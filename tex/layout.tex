% !TEX root =  thesis.tex

% This document describes all layout settings for page headings, section titles,
% linespread etcetera.

%----------------------------------------------------------------------------------------
%	CONTROL PACKAGES
%----------------------------------------------------------------------------------------

\usepackage[titletoc]{appendix}	% for additional appendix formatting commands
\usepackage[Bjornstrup]{fncychap}		% for chapter symbol formatting
\usepackage{titletoc}
\usepackage[nottoc]{tocbibind}	% Add the bibliography to the table of content
\usepackage{titlesec}
\usepackage{fancyhdr}
\usepackage[footnotesize,bf]{caption}	% change caption formatting

%----------------------------------------------------------------------------------------
%	TYPEFACE SETTINGS
%----------------------------------------------------------------------------------------

% A selection of font packages
%\usepackage{mathpazo}				% Adds palatino like math symbols
%\usepackage{helvet}				% Use helvetica as main fonts
%\usepackage[charter]{mathdesign} 	% Use 'Adobe Utopia' as math font
%\usepackage[garamond]{mathdesign} 	% Use 'URW Gara?mond' as math font
%\usepackage[utopia]{mathdesign} 	% Use 'Bit?stream Char?ter' as math font
%\usepackage{txfonts}				% Use 'Times'-line math font
%\newcommand\hmmax{0}\usepackage{bm}% Add bold font math symbols
%\usepackage[mathcal]{eucal}		% Use the 'Euler' math font

% Mix-n-match your preferred thesis font
\DeclareSymbolFont{usualmathcal}{OMS}{cmsy}{m}{n}			% set symbol font
\DeclareSymbolFontAlphabet{\mathcal}{usualmathcal}			% set calligraphic math symbols
\DeclareMathAlphabet{\mathbbm}{U}{bbm}{m}{n}				% set bold math font

%----------------------------------------------------------------------------------------
%	SPACING SETTINGS
%----------------------------------------------------------------------------------------

% Set spacing options
\linespread{1.1}
\addtolength{\footskip}{0.5cm}
\addtolength{\headheight}{\baselineskip}

\interfootnotelinepenalty=10000

%\addtolength{\headheight}{1.1pt}

%\setlength{\captionmargin}{4mm}

\renewcommand{\contentsname}{Contents}
\setcounter{tocdepth}{1}

\fnbelowfloat % place bottom float above the footnotes


%%%%%%%%%%%%%%%%%%%%%%%%
% FROM ROB

%\setlength{\parindent}{0cm}
\addtolength{\voffset}{1.5\baselineskip}
\addtolength{\evensidemargin}{-10mm}
\addtolength{\oddsidemargin}{10mm}

%\renewcommand{\sectfont}{\rmfamily\bfseries}

\renewcommand{\topfraction}{0.9}  % Florian: 0.9
\renewcommand{\bottomfraction}{0.9}  % Florian: 0.9
\renewcommand{\textfraction}{0.1}
\renewcommand{\floatpagefraction}{0.8}
\setlength{\floatsep}{14pt plus 10pt minus 4pt}
\setlength{\intextsep}{14pt plus 10pt minus 4pt}
\setlength{\textfloatsep}{20pt plus 12pt minus 4pt}

%\renewcommand{\figurename}{Figure}
%\renewcommand{\capfont}{\normalfont\small}
%\renewcommand*{\descfont}{\normalfont\bfseries}
%\renewcommand{\caplabelfont}{\bfseries}
%\setcapindent{0em}


%%%%%%%%%%%%%%%%%%%%%%%%%%%%%%%%%%%%%%%%%%%%%%%%%%%%%%%%%%%%%%%%%%%%%%%%%%%%%%%%
%  Here's where the tabletoc and titlesec stuff is put.
\definecolor{grijs}{gray}{0.7}%grijze kleur voor chapter nummer

%\titleformat{\chapter}[display]{}{\thechapter}{1pc}{}

%USE:
%\titleformat{cmd}[shape]{format}{label}{sep}{before-code}
%{cmd} = \chapter, \section etc.
%[shape]: 'display' puts the title chapter on a separate line, 'hang' a hanging label (label begint op zelfde regel als nummer en gaat eronder door), 'runin' should produce a run-in title, but that is not working, 'frame' puts a line-frame around the title.
%{format}: typeset, applies to both label and text
%{label}: formatting of the heading (chapter) number, refer to actual number by using \thechapter
%{sep}: distance between heading number and title text. Depending on the shape argument this can refer either to vertical or horizontal spacing
%{before-code}: code executed preceding the heading text, picks up heading text
%
%\titlespacing{cmd}{left-sep}{before-sep}{after-sep}[rigth-sep]
%{left-sep}: specifies the increase in the left margin
%{before-sep}: vertical space added above the heading
%{after-sep}: seperation between heading and following paragraph
%[rigth-sep]: specifying increase in right margin

%grijs nummer, zwarte vert. streep, titel hangend eronder:
%\titleformat{\chapter}[]
% {\normalfont\filcenter\sffamily}
%  { \raggedright  {\color{grijs}\fontsize{70}{70}\selectfont \thechapter}}
%    {0pc}
%      {\quad\rule[-12pt]{2pt}{70pt}\quad\huge }    

%AANGEPAST ORIGINEEL (kleinere letter, kleiner cijfer)
%\titleformat{\chapter}[display]
% {\normalfont\filcenter\sffamily}
% { \raggedleft   {\fontsize{70}{70}\selectfont \thechapter}}
%  {1pc}
%  {\titlerule
%  \vspace{1pc}
%   \huge }
      
%LEUK -- grijs cijfer (kleiner), grijze ruler
%\titleformat{\chapter}[display]
% {\normalfont\filcenter\sffamily}
% { \raggedleft   {\color{grijs}\fontsize{70}{70}\selectfont \thechapter}}%size of the chapter number
%  {1pc}
%  {{\color{grijs}\titlerule}
%  \vspace{1pc}
%   \huge }

%ORIGINEEL -- zwart cijfer
%\titleformat{\chapter}[display]
% {\normalfont\filcenter\sffamily}
% { \raggedleft   {\fontsize{90}{90}\selectfont \thechapter}}%size of the chapter number
%  {1pc}%afstand tussen lijn en cijfer
%  {\titlerule
%  \vspace{1pc}%afstand tussen titel en lijn
%   \Huge }%titel lettergrootte

\titleformat{\section}
 {\normalfont\Large\bfseries}{\thesection}{1em}{}

\titleformat{\subsection}
 {\normalfont\large\bfseries}{\thesubsection}{1em}{}

\titleformat{\subsubsection}
 {\normalfont\normalsize\bfseries}{\thesubsubsection}{1em}{}


%%%%%%%%%%%%%%%%%%%%%%%%%%%%%%%%%%%%%%%%%%%%%%%%%%%%%%%%%%%%%%%%%%%%%%%%%%%%%%%%

% This is to get completely empty pages on the left of a chapter opening page.
\makeatletter
\def\cleardoublepage{\clearpage\if@twoside \ifodd\c@page\else
  \hbox{}
  \thispagestyle{empty}
  \newpage
  \if@twocolumn\hbox{}\newpage\fi\fi\fi}
\makeatother

%%%%%%%%%%%%%%%%%%%%%%%%%%%%%%%%%%%%%%%%%%%%%%%%%%%%%%%%%%%%%%%%%%%%%%%%%%%%%%%%
%  fancyhdr settings

\pagestyle{fancy}


%THESE COMMANDS GET THE CHAPTER HEADER AND NUMBER
%\renewcommand{\chaptermark}[1]{\markboth{
%  \sffamily \small \bfseries \chaptertitlename\ \thechapter:\ #1}{}}
\renewcommand{\chaptermark}[1]{\markboth{
   \sffamily \small \slshape \thechapter \; \ #1 }{}}
\renewcommand{\headrulewidth}{0.5pt}

\renewcommand{\sectionmark}[1]{\markright{
  \sffamily \small \slshape  \thesection  \; \ #1 }}

% Chapter title on the left, paragraph on the right.
% Page numbers on the outside.
%\fancyhf[HLE]{\nouppercase{\leftmark}}
%\fancyhf[HRO]{\nouppercase{\rightmark}}
%\fancyhf[HLO]{\thepage}
%\fancyhf[HLE]{\thepage}
% -----
\fancyhf{}
%\fancyhead[LE]{\leftmark}
%\fancyhead[RO]{\rightmark}

\fancyhead[LE]{\nouppercase{\leftmark}}
\fancyhead[RO]{\nouppercase{\rightmark}}



\fancyfoot[LO,RE]{\small\thepage}
\fancyfoot[C]{}

% for the chapter opening pages:
\fancypagestyle{plain}{
  \fancyhf{}
  \fancyfoot[LO,RE]{\small\thepage}
  \fancyfoot[C]{}
  \renewcommand{\headrulewidth}{0pt}
}


\setlength{\parskip}{0pt}

%%%%%%%%%%%%%%%%%%%%%%%%%%%%%%%%%%%%%%%%%%%%%%%%%%%%%%%%%%%%%%%%%%%%%%%%%%%%%%%%

% Custom environment definitions
\newenvironment{abstract}
{\thispagestyle{empty}\begin{center}{\it\normalsize Abstract}\\ \end{center}}
{\clearpage}

\newcommand{\chauthors}[1]{\begin{center} {\rm\normalsize #1} \\ \end{center}}
\newcommand{\chjournal}[1]{\begin{center} {\it\normalsize #1} \\ \medskip \end{center}}

%%%%%%%%%%%%%%%%%%%%%%%%%%%%%%%%%%%%%%%%%%%%%%%%%%%%%%%%%%%%%%%%%%%%%%%%%%%%%%%%


%\rfoot{\setlength{\unitlength}{1mm}
%\begin{picture}(5,0)
%\put(5,0){\includegraphics[width=3cm]{ok/expo000\thepage.epsi}}
%\end{picture}}

%%%%%%%%%%%%%%%%%%%%%%%%%%%%%%%%%%%%%%%%%%%%%%%%%%%%%%%%%%%%%%%%%%%%%%%%%%%%%%%%


